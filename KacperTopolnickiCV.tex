\documentclass{article}
\usepackage{hyperref}
\usepackage{graphicx}

\providecommand{\tightlist}{%
  \setlength{\itemsep}{0pt}\setlength{\parskip}{0pt}}

\begin{document}

\hypertarget{section}{%
\section{}\label{section}}

\begin{itemize}
\tightlist
\item
  \protect\hyperlink{personal-details}{Personal details}
\item
  \protect\hyperlink{education}{Education}
\item
  \protect\hyperlink{positions}{Positions}
\item
  \protect\hyperlink{skills-and-experience}{Skills and experience}
\item
  \protect\hyperlink{grants}{Grants}
\item
  \href{./0pl.html}{Publications (WOS, I 2021)}
\item
  \href{./00pl.html}{Talks}
\item
  \protect\hyperlink{other-interests}{Other interests}
\end{itemize}

\hypertarget{personal-details}{%
\section{Personal details}\label{personal-details}}

\begin{itemize}
\tightlist
\item
  \emph{e-mail}:
  \href{mailto:kacpertopol@gmail.com}{\nolinkurl{kacpertopol@gmail.com}}
\item
  \emph{cell phone number}: (+48) 728 364 517
\end{itemize}

\hypertarget{education}{%
\section{Education}\label{education}}

\begin{itemize}
\tightlist
\item
  \textbf{2014}

  \begin{itemize}
  \tightlist
  \item
    Ph.D.~- Physics

    \begin{itemize}
    \tightlist
    \item
      \emph{institution}: Faculty of Physics, Astronomy and Applied
      Computer Science of the Jagiellonian University
    \item
      \emph{description}: My thesis titled: ``The two-nucleon and
      three-nucleon systems in three dimensions'' was successfully
      defended on the 23 of September 2014. Subsequently I was awarded a
      Ph.D. degree in physics on 25 September 2014 after the positive
      decision of the Faculty of Physics, Astronomy and Applied Computer
      Science council.
    \end{itemize}
  \end{itemize}
\item
  \textbf{2011-2012}

  \begin{itemize}
  \tightlist
  \item
    \(\approx 0.5\) year Internship at the Institute of Nuclear Physics
    in Jülich

    \begin{itemize}
    \tightlist
    \item
      \emph{institution}: Institut für Kernphysik, Forschungszentrum
      Jülich, Germany
    \item
      \emph{description}: My half year stay at the interdisciplinary
      research center in Jülich gave me the possibility to have direct
      access to specialists in parallel computing on large computing
      clusters. This was a great opportunity and had a big impact on my
      research.
    \end{itemize}
  \end{itemize}
\item
  \textbf{2010-2014}

  \begin{itemize}
  \tightlist
  \item
    International PhD Studies

    \begin{itemize}
    \tightlist
    \item
      \emph{institution}: Faculty of Physics, Astronomy and Applied
      Computer Science of the Jagiellonian University
    \item
      \emph{description}: On my second year of regular PhD studies I
      switched over to the
      \href{https://fais.uj.edu.pl/applied-nuclear-physics-and-innovative-technologies}{International
      PhD Studies in Applied Nuclear Physics and Innovative
      Technologies}. The studies lasted for four years. My chosen topic
      ``Few-nucleon fusion reactions'' was being supervised of professor
      Jacek Golak.
    \end{itemize}
  \end{itemize}
\item
  \textbf{2009-2014}

  \begin{itemize}
  \tightlist
  \item
    Ph.D.~studies in Physics

    \begin{itemize}
    \tightlist
    \item
      \emph{institution}: Faculty of Physics, Astronomy and Applied
      Computer Science of the Jagiellonian University
    \item
      \emph{description}: PhD studies in Nuclear Physics under the
      supervision of professor Jacek Golak. My research was focused on
      the description of few (two, three) nucleon systems. The character
      of my work was theoretical and included the development of
      effective implementations of numerical calculations on large
      computing clusters (JUQUEEN at the Jülich Supercomputing Centre).
      A large portion of my efforts were focused on developing
      consistent methods for treating the complicated algebraic
      expressions that arise in the calculations. The development of
      these methods was made possible with the use of symbolic
      programming within the
      \href{https://www.wolfram.com/mathematica/}{Mathematica} system.
    \end{itemize}
  \end{itemize}
\item
  \textbf{2009}

  \begin{itemize}
  \tightlist
  \item
    M.Sc - Physics

    \begin{itemize}
    \tightlist
    \item
      \emph{institution}: Faculty of Physics, Astronomy and Applied
      Computer Science of the Jagiellonian University
    \item
      \emph{description}: In 2009 I finished my five year studies in
      physics (majored in theoretical physics). My masters thesis
      titled: ``Lattice models of chiral liquid crystal phases in Monte
      Carlo simulations'' was written under the supervision of professor
      Lech Longa.
    \end{itemize}
  \end{itemize}
\item
  \textbf{2004-2009}

  \begin{itemize}
  \tightlist
  \item
    Studies in Physics

    \begin{itemize}
    \tightlist
    \item
      \emph{description}: Faculty of Physics, Astronomy and Applied
      Computer Science of the Jagiellonian University
    \item
      \emph{institution}: In 2004 I started ``Studies in Mathematics and
      Natural Sciences''. I chose to specialize in theoretical physics.
    \end{itemize}
  \end{itemize}
\end{itemize}

\hypertarget{positions}{%
\section{Positions}\label{positions}}

\begin{itemize}
\tightlist
\item
  \textbf{2020-}

  \begin{itemize}
  \tightlist
  \item
    {[}current position{]} adiunkt (english equivalent: assistant
    professor) at the Institute of Physics, Jagiellonian University,
    Kraków, Poland
  \end{itemize}
\item
  \textbf{2017-2020}

  \begin{itemize}
  \tightlist
  \item
    {[}3 years{]} asystent naukowy (english equivalent: research
    assistant) at the Institute of Physics, Jagiellonian University,
    Kraków, Poland
  \end{itemize}
\item
  \textbf{2014-2017}

  \begin{itemize}
  \tightlist
  \item
    {[}2015, \(\approx\) 0.5 year{]} post-doc at Texas A\&M University
    Commerce
  \item
    various grants (more information in \textbf{Grants} section)
  \end{itemize}
\item
  \textbf{2009-2014}

  \begin{itemize}
  \tightlist
  \item
    {[}2009-2014, 5 years{]} PhD student at the Institute of Physics
    (nuclear theory, few body systems), Jagiellonian University, Kraków,
    Poland
  \item
    {[}2010-2014, 4 years{]} stipend from the
    \href{https://fais.uj.edu.pl/applied-nuclear-physics-and-innovative-technologies}{International
    PhD Studies in Applied Nuclear Physics and Innovative Technologies}
  \item
    {[}2011-2012, \(\approx\) 0.5 year{]} internship at the
    \href{https://www.fz-juelich.de/ikp/EN/Home/home_node.html}{Nuclear
    Physics Institute (IKP)} in Forschungszentrum Jülich, Germany
  \end{itemize}
\end{itemize}

\hypertarget{skills-and-experience}{%
\section{Skills and experience}\label{skills-and-experience}}

\begin{itemize}
\tightlist
\item
  programming

  \begin{itemize}
  \tightlist
  \item
    languages

    \begin{itemize}
    \tightlist
    \item
      experienced: Wolfram Language, Fortran, python, bash
    \item
      literate: c, c++, haskell
    \item
      learning: rust
    \end{itemize}
  \item
    parallel computing on large computing clusters using MPI, OPENMP

    \begin{itemize}
    \tightlist
    \item
      worked with multiple generations of supercomputers located at the
      \href{https://www.fz-juelich.de/en/ias/jsc/systems/supercomputers}{Juelich
      Supercomputing Centre}
    \end{itemize}
  \item
    worked with the \href{https://gitlab.cern.ch/atlas/athena}{athena}
    codebase

    \begin{itemize}
    \tightlist
    \item
      part of my ATLAS \emph{Qualification Task}
    \end{itemize}
  \item
    \href{https://kacpertopol.github.io/betterCamera/}{betterCamera}

    \begin{itemize}
    \tightlist
    \item
      useful python tool for teaching during the pandemic lockdowns
    \item
      succesor to my previous popular project:
      \href{https://github.com/kacpertopol/cam_board}{cam\_board}
    \end{itemize}
  \item
    Monads and \emph{do} notation in the Wolfram Language

    \begin{itemize}
    \tightlist
    \item
      introduce Haskell style \emph{do} notation and monads to
      Mathematica
    \item
      project available on
      \href{https://gitlab.com/kacpertopolnicki/wlmonad}{GitLab},
      description available on
      \href{https://arxiv.org/abs/2005.09478}{arxiv}
    \end{itemize}
  \end{itemize}
\item
  communication

  \begin{itemize}
  \tightlist
  \item
    Polish (native)
  \item
    English (CAE - 2003, CPE - 2006)
  \end{itemize}
\item
  academic

  \begin{itemize}
  \tightlist
  \item
    few body systems
  \item
    nuclear theory
  \item
    numerical calculations
  \item
    data analysis for detector trigger systems
  \item
    teaching
  \end{itemize}
\item
  software

  \begin{itemize}
  \tightlist
  \item
    standard linux tools: vim, ssh, git, make, etc.
  \item
    Mathematica (symbolic programming, mathematics)
  \item
    \href{https://openscad.org/}{openSCAD} (programatic CAD)
  \item
  \end{itemize}
\end{itemize}

\hypertarget{grants}{%
\section{Grants}\label{grants}}

\begin{itemize}
\tightlist
\item
  principal investigator

  \begin{itemize}
  \tightlist
  \item
    {[}2017-2020, 3 years{]} SONATA 11

    \begin{itemize}
    \tightlist
    \item
      \emph{funding agency}: National Science Center, Poland
    \item
      \emph{grant number}: DEC-2016/21/D/ST2/01120
    \item
      \emph{title}: ``Development of analytical and numerical
      computation techniques related to few-nucleon systems''
    \end{itemize}
  \item
    {[}2014-2016, 2 years{]} PRELUDIUM 6

    \begin{itemize}
    \tightlist
    \item
      \emph{funding agency}: National Science Center, Poland
    \item
      \emph{grant number}: DEC-2013/11/N/ST2/03733
    \item
      \emph{title}: ``Development of techniques using a three
      dimensional representation of nucleonic degrees of freedom in
      few-nucleon bound and scattering state calculations''
    \end{itemize}
  \end{itemize}
\item
  co-executor

  \begin{itemize}
  \tightlist
  \item
    {[}2017{]} HARMONIA 8

    \begin{itemize}
    \tightlist
    \item
      \emph{funding agency}: National Science Center, Poland
    \item
      \emph{grant number}: DEC-2016/22/M/ST2/00173
    \item
      \emph{title}: ``Utilizing consistent chiral nuclear potentials and
      electroweak currents in order to describe three nucleon reactions
      and properties of nuclei''
    \end{itemize}
  \item
    {[}2014-2016{]} HARMONIA 5

    \begin{itemize}
    \tightlist
    \item
      \emph{funding agency}: National Science Center, Poland
    \item
      \emph{grant number}: DEC-2013/10/M/ST2/00420
    \item
      \emph{title}: ``Investigation of the properties of light nuclei
      and three body processes based on chiral nuclear potentials)''
    \end{itemize}
  \end{itemize}
\end{itemize}

\hypertarget{publications-wos-i-2021}{%
\section{\texorpdfstring{\href{./0pl.html}{Publications (WOS, I
2021)}}{Publications (WOS, I 2021)}}\label{publications-wos-i-2021}}

\hypertarget{talks}{%
\section{\texorpdfstring{\href{./00pl.html}{Talks}}{Talks}}\label{talks}}

\hypertarget{other-interests}{%
\section{Other interests}\label{other-interests}}

\begin{itemize}
\tightlist
\item
  \emph{blog}: \url{https://kacpertopol.github.io/myblog/}
\item
  \emph{music}: Tom Waits, Gelnn Gould
\item
  \emph{sport}: skiing, running
\end{itemize}
\end{document}
